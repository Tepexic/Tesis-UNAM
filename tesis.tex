%%%%%%%%%%%%%%%%%%%%%%%%%%%%%%%%%%%%%%%%%%%%%%%%%%%%%%%%%%%%%%%%%%%%%%%%%%%%%%%%
%                         FORMATO DE TESIS UMSNH                               %
%%%%%%%%%%%%%%%%%%%%%%%%%%%%%%%%%%%%%%%%%%%%%%%%%%%%%%%%%%%%%%%%%%%%%%%%%%%%%%%%
% based on Harish Bhanderi's PhD/MPhil template, then Uni Cambridge
% http://www-h.eng.cam.ac.uk/help/tpl/textprocessing/ThesisStyle/
% corrected and extended in 2007 by Jakob Suckale, then MPI-iCBG PhD programme
% and made available through OpenWetWare.org - the free biology wiki

%                     Under GNU License v3
% Adaptado para UNAM: @Tepexic
% ADAPTADO PARA UMSNH:  @arturolp

\documentclass[twoside,11pt]{Latex/Classes/PhDthesisPSnPDF} % "thesisUMSNH" para formato de la UMSNH
%         PUEDEN INCLUIR EN ESTE ESPACIO LOS PAQUETES EXTRA, O BIEN, EN EL ARCHIVO "PhDthesisPSnPDF.cls" EN "./Latex/Classes/"

% Estos paquetes son opcionales y a necesidad de cada quien:
\usepackage{blindtext}                             % Para insertar texto dummy, de ejemplo, pues.
\usepackage{amssymb, amsmath, amsbsy, amsfonts}    % ECUACIONES Y SÍMBOLOS MATEMÁTICOS
\usepackage{listings}                    % PERMITE AGREGAR CÓDIGO DE LENGUAJES  DE PROGRAMACIÓN (DOCUMENTACIÓN EN GOOGLE)
\usepackage{mathdots}                    % para el comando \iddots
\usepackage{mathrsfs}                    % para formato de letra en ecuaciones
\usepackage[round, sort, numbers]{natbib}  % Personalizar la bibliografía a gusto de cada quien

% Note:
% The \blindtext or \Blindtext commands throughout this template generate dummy text
% to fill the template out. These commands should all be removed when 
% writing thesis content.


% This file contains macros that can be called up from connected TeX files
% It helps to summarise repeated code, e.g. figure insertion (see below).

%%%%%%%%%%%%%%%%%%%%%%%%%%%%%%%%%%%%%%%%%%%%%%
%            Colores de la UNAM              %
%%%%%%%%%%%%%%%%%%%%%%%%%%%%%%%%%%%%%%%%%%%%%%
%Azul Pantone 541  -->(0,63,119) RGB
\definecolor{Azul}{RGB}{0,63,119}

%Oro Pantone 460  -->(234,221,150) RGB
\definecolor{Oro}{RGB}{234,221,150}


%%%%%%%%%%%%%%%%%%%%%%%%%%%%%%%%%%%%%%%%%%%%%%
%            Comandos para líneas            %
%%%%%%%%%%%%%%%%%%%%%%%%%%%%%%%%%%%%%%%%%%%%%%
%Se define un comando \colorvrule para hacer líneas verticales de color con 3 argumentos: color, ancho, alto
\newcommand{\colorvrule}[3]{
\begingroup\color{#1}\vrule width#2 height#3
\endgroup}

%Se define un comando \colorhrule para hacer líneas horizontales de color con 2 argumentos: color, ancho
\newcommand{\colorhrule}[2]{
\begingroup\color{#1}\hrule height#2
\endgroup}

%%%%%%%%%%%%%%%%%%%%%%%%%%%%%%%%%%%%%%%%%%%%%%
%          Comando para derivadas            %
%%%%%%%%%%%%%%%%%%%%%%%%%%%%%%%%%%%%%%%%%%%%%%
\newcommand{\derivada}[3][]{\ensuremath{\dfrac{\mbox{d}^{#1}#2}{\mbox{d}#3^{#1}}}} 
%primer argumento(opcional): orden de la derivada
%segundo argumento: función a derivar
%tercer argumento: variable respecto a la que se deriva


%%%%%%%%%%%%%%%%%%%%%%%%%%%%%%%%%%%%%%%%%%%%%%
%       Comando para la exponencial          %
%%%%%%%%%%%%%%%%%%%%%%%%%%%%%%%%%%%%%%%%%%%%%%
\newcommand{\e}[1][]{\ensuremath{\mbox{e}^{#1}}}
%primer argumento(opcional): exponente de la exponencial




% insert a centered figure with caption and description
% parameters 1:filename, 2:title, 3:description and label
\newcommand{\figuremacro}[3]{
	\begin{figure}[htbp]
		\centering
		\includegraphics[width=1\textwidth]{#1}
		\caption[#2]{\textbf{#2} - #3}
		\label{condicion}
	\end{figure}
}

% insert a centered figure with caption and description AND WIDTH
% parameters 1:filename, 2:title, 3:description and label, 4: textwidth
% textwidth 1 means as text, 0.5 means half the width of the text
\newcommand{\figuremacroW}[4]{
	\begin{figure}[htbp]
		\centering
		\includegraphics[width=#4\textwidth]{#1}
		\caption[#2]{\textbf{#2} - #3}
		\label{#1}
	\end{figure}
}

% inserts a figure with wrapped around text; only suitable for NARROW figs
% o is for outside on a double paged document; others: l, r, i(inside)
% text and figure will each be half of the document width
% note: long captions often crash with adjacent content; take care
% in general: above 2 macro produce more reliable layout
\newcommand{\figuremacroN}[3]{
	\begin{wrapfigure}{o}{0.5\textwidth}
		\centering
		\includegraphics[width=0.48\textwidth]{#1}
		\caption[#2]{{\small\textbf{#2} - #3}}
		\label{#1}
	\end{wrapfigure}
}

% predefined commands by Harish
\newcommand{\PdfPsText}[2]{
  \ifpdf
     #1
  \else
     #2
  \fi
}

\newcommand{\IncludeGraphicsH}[3]{
  \PdfPsText{\includegraphics[height=#2]{#1}}{\includegraphics[bb = #3, height=#2]{#1}}
}

\newcommand{\IncludeGraphicsW}[3]{
  \PdfPsText{\includegraphics[width=#2]{#1}}{\includegraphics[bb = #3, width=#2]{#1}}
}

\newcommand{\InsertFig}[3]{
  \begin{figure}[!htbp]
    \begin{center}
      \leavevmode
      #1
      \caption{#2}
      \label{#3}
    \end{center}
  \end{figure}
}







%%% Local Variables:
%%% mode: latex
%%% TeX-master: "~/Documents/LaTeX/CUEDThesisPSnPDF/thesis"
%%% End:
           % Archivo con funciones útiles





%%%%%%%%%%%%%%%%%%%%%%%%%%%%%%%%%%%%%%%%%%%%%%%%%%%%%%%%%%%%%%%%%%%%%%%%%%%%%%%%
%                                   DATOS                                      %
%%%%%%%%%%%%%%%%%%%%%%%%%%%%%%%%%%%%%%%%%%%%%%%%%%%%%%%%%%%%%%%%%%%%%%%%%%%%%%%%
\title{Título de la tesis}
\author{Nombre Apellido1 Apellido 2} 
\facultad{Facultad de Algo Seguramente Muy Importante}                 % Nombre de la facultad/escuela
\escudofacultad{Latex/Classes/Escudos/fi_azul} % Aquí ponen la ruta y nombre del escudo de su facultad, actualmente, la carpeta Latex/Classes/Escudos cuenta con los siguientes escudos:
% "fi_azul" Facultad de ingenieria en color azul
% "fi_negro" Facultad de ingenieria en color negro
% "fc_azul" Facultad de ciencias en color azul
% "fc_negro" Facultad de ciencias en color negro
% "fmed_grande" Facultad de medicina UMSNH
% Se agradecen sus aportaciones de escudos a jebus.velazquez@gmail.com

\degree{Médic@/Ingenier@/Licenciad@}       % Carrera
\director{Nombre Director}                 % Director de tesis
%\tutor{Nombre  Tutor }                    % Tutor de tesis, si aplica
\degreedate{2020}                          % Año de la fecha del examen
\lugar{Ciudad Universitaria, CDMX}         % Lugar

%\portadafalse                              % Portada en NEGRO, descomentar y comentar la línea siguiente si se quiere utilizar
\portadatrue                                % Portada en COLOR



%% Opciones del posgrado (descomentar si las necesitan)
	%\posgradotrue                                                    
	%\programa{programa de maestría y doctorado en ingeniería}
	%\campo{Ingeniería Eléctrica - Control}
	%% En caso de que haya comité tutor
	%\comitetrue
	%\ctutoruno{Dr. Emmet L. Brown}
	%\ctutordos{Dr. El Doctor}
%% Datos del jurado                             
	%\presidente{Dr. 1}
	%\secretario{Dr. 2}
	%\vocal{Dr. 3}
	%\supuno{Dr. 4}
	%\supdos{Dr. 5}
	%\institucion{el Instituto de Ingeniería, UNAM}

\keywords{tesis,autor,tutor,etc}            % Palablas clave para los metadatos del PDF
\subject{tema_1,tema_2}                     % Tema para metadatos del PDF  

%%%%%%%%%%%%%%%%%%%%%%%%%%%%%%%%%%%%%%%%%%%%%%%%%%%%%
%                   PORTADA                         %
%%%%%%%%%%%%%%%%%%%%%%%%%%%%%%%%%%%%%%%%%%%%%%%%%%%%%
\begin{document}

\maketitle									% Se redefinió este comando en el archivo de la clase para generar automáticamente la portada a partir de los datos

%%%%%%%%%%%%%%%%%%%%%%%%%%%%%%%%%%%%%%%%%%%%%%%%%%%%%
%                  PRÓLOGO                          %
%%%%%%%%%%%%%%%%%%%%%%%%%%%%%%%%%%%%%%%%%%%%%%%%%%%%%
\frontmatter
\begin{dedication}
A la Facultad de Ingeniería y a la  Universidad, por la formación que me han dado.\\
Es gracias a ustedes que es posible el presente trabajo.\\
En verdad, gracias.\\
Yo.
\end{dedication}
       % Comentar línea si no se usa
%\chapter*{}
%\pagenumbering{Roman}

\begin{acknowledgements}

También quisiera reconocer a ... por ...CONACYT,  PAPIIT / etc.
\blindtext % Dummy text
\end{acknowledgements}




   % Comentar línea si no se usa 
% ******************************* Thesis Declaration ********************************

\begin{declaration}

Por la presente declaro que, salvo cuando se haga referencia específica al trabajo de otras personas, el contenido de esta tesis es original y no se ha presentado total o parcialmente para su consideración para cualquier otro título o grado en esta o cualquier otra Universidad. Esta tesis es resultado de mi propio trabajo y no incluye nada que sea el resultado de algún trabajo realizado en colaboración, salvo que se indique específicamente en el texto. 
% Author and date will be inserted automatically from thesis.tex


\end{declaration}
           % Comentar línea si no se usa

% Thesis Abstract -----------------------------------------------------


%\begin{abstractslong}    %uncommenting this line, gives a different abstract heading
\begin{abstracts} 
Los sistemas de comunicación ópticos representan en la actualidad una de las tecnologías más empleadas en el sector de las telecomunicaciones, dada su eficiencia, velocidad y demás ventajas provistas por éste desarrollo. Desde una visión esencialista, un sistema de comunicación óptico se compone por un transmisor, un medio de transmisión (Principalmente fibra óptica) y un receptor.

La adecuación de señales eléctricas al medio óptico es en términos generales, la principal tarea ejecutada por un transmisor siguiendo el paradigma actual; además, éste principio no limita sus aplicaciones a un sistema de telecomunicaciones, sino también a sistemas de medición, reflectómetros y métodos de monitoreo y sensado.

El presente trabajo se ocupa del diseño, análisis, desarrollo e implementación de un sistema transmisor óptico, desde la generación de una señal eléctrica hasta su adecuación al medio óptico.   

\end{abstracts}
%\end{abstractlongs}


% ----------------------------------------------------------------------                   % Comentar línea si no se usa

%%%%%%%%%%%%%%%%%%%%%%%%%%%%%%%%%%%%%%%%%%%%%%%%%%%%%
%                   ÍNDICES                         %
%%%%%%%%%%%%%%%%%%%%%%%%%%%%%%%%%%%%%%%%%%%%%%%%%%%%%
%Esta sección genera el índice
\setcounter{secnumdepth}{3} % organisational level that receives a numbers
\setcounter{tocdepth}{3}    % print table of contents for level 3
\tableofcontents            % Genera el índice 
%: ----------------------- list of figures/tables ------------------------
\listoffigures              % Genera el ínidce de figuras, comentar línea si no se usa
\listoftables               % Genera índice de tablas, comentar línea si no se usa


%%%%%%%%%%%%%%%%%%%%%%%%%%%%%%%%%%%%%%%%%%%%%%%%%%%%%
%                   CONTENIDO                       %
%%%%%%%%%%%%%%%%%%%%%%%%%%%%%%%%%%%%%%%%%%%%%%%%%%%%%
% the main text starts here with the introduction, 1st chapter,...
\mainmatter
\def\baselinestretch{1.5}                   % Interlineado de 1.5

% this file is called up by thesis.tex
% content in this file will be fed into the main document
%----------------------- introduction file header -----------------------
%%%%%%%%%%%%%%%%%%%%%%%%%%%%%%%%%%%%%%%%%%%%%%%%%%%%%%%%%%%%%%%%%%%%%%%%%
%  Capítulo 1: Introducción- DEFINIR OBJETIVOS DE LA TESIS              %
%%%%%%%%%%%%%%%%%%%%%%%%%%%%%%%%%%%%%%%%%%%%%%%%%%%%%%%%%%%%%%%%%%%%%%%%%

\chapter{Introducción}

%: ----------------------- HELP: latex document organisation
% the commands below help you to subdivide and organise your thesis
%    \chapter{}       = level 1, top level
%    \section{}       = level 2
%    \subsection{}    = level 3
%    \subsubsection{} = level 4
%%%%%%%%%%%%%%%%%%%%%%%%%%%%%%%%%%%%%%%%%%%%%%%%%%%%%%%%%%%%%%%%%%%%%%%%%
%                           Presentación                                %
%%%%%%%%%%%%%%%%%%%%%%%%%%%%%%%%%%%%%%%%%%%%%%%%%%%%%%%%%%%%%%%%%%%%%%%%%

\section{Presentación} % section headings are printed smaller than chapter names
\blindtext
%%%%%%%%%%%%%%%%%%%%%%%%%%%%%%%%%%%%%%%%%%%%%%%%%%%%%%%%%%%%%%%%%%%%%%%%%
%                           Objetivo                                    %
%%%%%%%%%%%%%%%%%%%%%%%%%%%%%%%%%%%%%%%%%%%%%%%%%%%%%%%%%%%%%%%%%%%%%%%%%

\section{Objetivo}

Este trabajo tiene por objetivo ...
\blindtext
%%%%%%%%%%%%%%%%%%%%%%%%%%%%%%%%%%%%%%%%%%%%%%%%%%%%%%%%%%%%%%%%%%%%%%%%%
%                           Motivación y estado del arte                %
%%%%%%%%%%%%%%%%%%%%%%%%%%%%%%%%%%%%%%%%%%%%%%%%%%%%%%%%%%%%%%%%%%%%%%%%%
\section{Motivación}


\blindtext
%%%%%%%%%%%%%%%%%%%%%%%%%%%%%%%%%%%%%%%%%%%%%%%%%%%%%%%%%%%%%%%%%%%%%%%%%
%                   Planteamiento del problema                          %
%%%%%%%%%%%%%%%%%%%%%%%%%%%%%%%%%%%%%%%%%%%%%%%%%%%%%%%%%%%%%%%%%%%%%%%%%

\section{Planteamiento del problema}
\blindtext
%%%%%%%%%%%%%%%%%%%%%%%%%%%%%%%%%%%%%%%%%%%%%%%%%%%%%%%%%%%%%%%%%%%%%%%%%
%                           Metodología                                 %
%%%%%%%%%%%%%%%%%%%%%%%%%%%%%%%%%%%%%%%%%%%%%%%%%%%%%%%%%%%%%%%%%%%%%%%%%
\section{Metodología}

Se tiene un objetivo principal, y para llegar a \'el %otra forma de poenr acentos
\blindtext
%%%%%%%%%%%%%%%%%%%%%%%%%%%%%%%%%%%%%%%%%%%%%%%%%%%%%%%%%%%%%%%%%%%%%%%%%
%                         Contribuciones                                %
%%%%%%%%%%%%%%%%%%%%%%%%%%%%%%%%%%%%%%%%%%%%%%%%%%%%%%%%%%%%%%%%%%%%%%%%%

\section{Contribuciones}

La principal contribución de este trabajo es 
\blindtext
%%%%%%%%%%%%%%%%%%%%%%%%%%%%%%%%%%%%%%%%%%%%%%%%%%%%%%%%%%%%%%%%%%%%%%%%%
%                           Estructura de la tesis                      %
%%%%%%%%%%%%%%%%%%%%%%%%%%%%%%%%%%%%%%%%%%%%%%%%%%%%%%%%%%%%%%%%%%%%%%%%%

\section{Estructura de la tesis}

Este trabajo está dividido en XX capítulos. Al principio se encuentra 
\\\\
Finalmente se encuentra la parte de             % ~10 páginas - Explicar el propósito de la tesis

%%%%%%%%%%%%%%%%%%%%%%%%%%%%%%%%%%%%%%%%%%%%%%%%%%%%%%%%%%%%%%%%%%%%%%%%%
%           Capítulo 2: MARCO TEÓRICO - REVISIÓN DE LITERATURA
%%%%%%%%%%%%%%%%%%%%%%%%%%%%%%%%%%%%%%%%%%%%%%%%%%%%%%%%%%%%%%%%%%%%%%%%%

\chapter{Marco teórico}
En este capítulo, normalmete se ponen todas las ecuaciones que se van a usar en la tesis, así ya nomás se hace rferencia a la ecuación tal o "como se vió en el capítulo 2", y esas cosas.
%inserción de codigo de Matlab
%Es conveniente sangrarlo (los de proteco dicen "indentarlo") para que no se encime con los números  de las líneas a la izquierda
\begin{lstlisting}[frame=single]
    % Declaracion de las variables simbolicas
    syms u z1 z2 z3 z4 J m M g l 
    % Matrices involucradas
    E = [J+m*l*l m*l*cos(z1);m*l*cos(z1) M+m] 
    F = [m*g*l*sin(z1);u+m*l*(z3*z3)*sin(z1)] 
    % Despeje
    V = E\F
\end{lstlisting}

\blindtext           % ~20 páginas - Poner un contexto a la tesis, hacer referencia a trabajos actuales en el tema

%%%%%%%%%%%%%%%%%%%%%%%%%%%%%%%%%%%%%%%%%%%%%%%%%%%%%%%%%%%%%%%%%%%%%%%%%
%           Capítulo 3: NOMBRE                   %
%%%%%%%%%%%%%%%%%%%%%%%%%%%%%%%%%%%%%%%%%%%%%%%%%%%%%%%%%%%%%%%%%%%%%%%%%

\chapter{Diseño del experimento}
En este capítulo, se presenta la introducción al desarrollo de la tesis, ya sea el modelo matemático o las bases del proyecto, etc.
Ejemplo de cita  [\citet{latex}]
Ejemplo de cita [\citeauthor{RR73}]
 % The \cite command functions as follows:
 %   \citet{key} ==>>                Jones et al. (1990)
 %   \citet*{key} ==>>               Jones, Baker, and Smith (1990)
 %   \citep{key} ==>>                (Jones et al., 1990)
 %   \citep*{key} ==>>               (Jones, Baker, and Smith, 1990)
 %   \citep[chap. 2]{key} ==>>       (Jones et al., 1990, chap. 2)
 %   \citep[e.g.][]{key} ==>>        (e.g. Jones et al., 1990)
 %   \citep[e.g.][p. 32]{key} ==>>   (e.g. Jones et al., p. 32)
 %   \citeauthor{key} ==>>           Jones et al.
 %   \citeauthor*{key} ==>>          Jones, Baker, and Smith
 %   \citeyear{key} ==>>             1990





%%%%%%%%%%%%%%%%%%%%%%%%%%%%%%%%%%%%%%%%%%%%%%%%%%%%%%%%%%%%%%%%%%%%%%%%%
%                          Descripción de la planta                     %
%%%%%%%%%%%%%%%%%%%%%%%%%%%%%%%%%%%%%%%%%%%%%%%%%%%%%%%%%%%%%%%%%%%%%%%%%
\section{Sección}
El sistema blah, blah. Ejemplo de cita \citep{texbook}
La figura (\ref{planta})                     %hace referencia a la imagen "planta" el número se inserta automáticamente
 ilustra los componentes de la planta.

\begin{figure}
  \centering
    \includegraphics[scale=0.5]{Capitulo3/figs/planta.jpg}      %Ruta completa de la imagen, porque se compila desde el archivo tesis.tex
  \caption{Descripción de la planta}            %Pie de imagen
  \label{planta}                            %nombre de referencia
\end{figure}




%%%%%%%%%%%%%%%%%%%%%%%%%%%%%%%%%%%%%%%%%%%%%%%%%%%%%%%%%%%%%%%%%%%%%%%%%
%                          Modelado                                     %
%%%%%%%%%%%%%%%%%%%%%%%%%%%%%%%%%%%%%%%%%%%%%%%%%%%%%%%%%%%%%%%%%%%%%%%%%
\section{\textcolor{Azul}{Sección en color azul}}
\subsection{Subsección}
Antes de comenzar, se definen  en la tabla ~\ref{tab:tabla} los parámetros y variables utilizadas

%%%%%%%%Tabla Nombres de parámetros
\begin{table}[htdp]                             %Inicia el entorno table debajo del texto
\centering\                                     %   centra la tabla
\begin{tabular}{||c | c ||}                     %inicia entorno tabular con doble línea en las orillas, 2 columnas con el contenido centrado (c)
\hline                                          %inserta línea horizontal
\hline
Nombre Parámetro/Variable & Símbolo\\
\hline
\hline
Masa del péndulo & $m$ \\
\hline
Masa del carro & $M$\\
\hline
Distancia del eje de giro al centro de masa & $l$ \\
\hline
Aceleración gravitatoria & $g$ \\
\hline
Momento de inercia péndulo respecto del eje de giro& $J$ \\
\hline
Ángulo del péndulo respecto del eje vertical & $\theta$\\
\hline
Velocidad angular del péndulo & $\dot{\theta}$, $\omega$\\
\hline
Distancia del carro respecto al centro del riel & x\\
\hline
Velocidad del carro & $\dot{x}$, $v$\\
\hline
\hline
\end{tabular}
\caption[Parámetros dinámicos del carro-péndulo]{\textbf{Parámetros dinámicos del carro-péndulo} - Estos son los valores de parámetros utilizados en el diseño y las simulaciones, corresponden a los valores reales.}
\label{tab:tabla}                              %etiqueta para referencia
\end{table}

\blindtext


%%%%%%%%%%%%%%%%%%%%%%%%%%%%%%%%%%%%%%%%%%%%%%%%%%%%%%%%%%%%%%%%%%%%%%%%%
%                          Subsección
%%%%%%%%%%%%%%%%%%%%%%%%%%%%%%%%%%%%%%%%%%%%%%%%%%%%%%%%%%%%%%%%%%%%%%%%%

\subsection{Otra subsección}

\Blindtext      % ~20 páginas - Explicar el problema en específico que se va a resolver, la metodología y experimentos/métodos utilizados
\chapter{Análisis de Resultados}
\section{Resultados}
\Blindtext   % ~20 páginas - Presentar los resultados tal cual son, y analizarlos.
\include{Capitulo5/conclusiones}            % ~5 páginas - Resumir lo que se hizo y lo que no y comentar trabajos futuros sobre el tema

%%%%%%%%%%%%%%%%%%%%%%%%%%%%%%%%%%%%%%%%%%%%%%%%%%%%%
%                   APÉNDICES                       %
%%%%%%%%%%%%%%%%%%%%%%%%%%%%%%%%%%%%%%%%%%%%%%%%%%%%%
\appendix
% this file is called up by thesis.tex
% content in this file will be fed into the main document
\chapter{Código/Manuales/Publicaciones}
% top level followed by section, subsection

\section{Apéndice}

Apéndice
               % Colocar los circuitos, manuales, código fuente, pruebas de teoremas, etc.

%%%%%%%%%%%%%%%%%%%%%%%%%%%%%%%%%%%%%%%%%%%%%%%%%%%%%
%                   REFERENCIAS                     %
%%%%%%%%%%%%%%%%%%%%%%%%%%%%%%%%%%%%%%%%%%%%%%%%%%%%%
% existen varios estilos de bilbiografía, pueden cambiarlos a placer
\bibliographystyle{apalike} % otros estilos pueden ser abbrv, acm, alpha, apalike, ieeetr, plain, siam, unsrt

%El formato trae otros estilos, o pueden agregar uno que les guste:
%\bibliographystyle{Latex/Classes/PhDbiblio-case} % title forced lower case
%\bibliographystyle{Latex/Classes/PhDbiblio-bold} % title as in bibtex but bold
%\bibliographystyle{Latex/Classes/PhDbiblio-url} % bold + www link if provided
%\bibliographystyle{Latex/Classes/jmb} % calls style file jmb.bst

\bibliography{Bibliografia/referencias}             % Archivo .bib


\end{document}
